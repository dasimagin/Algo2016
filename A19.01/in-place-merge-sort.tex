\documentclass[11pt, oneside]{article}
\usepackage{amsmath}
\usepackage{latexsym}
\usepackage{geometry}
\geometry{letterpaper}
\usepackage[parfill]{parskip}
\usepackage{amssymb}

\usepackage[T2A]{fontenc}
\usepackage[utf8]{inputenc}
\usepackage[russian]{babel}

\parindent=1.25cm

\title{In-place merge sort}

\begin{document}
\maketitle

Разберем алгоритм сортировки merge sort, требующий лишь константной памяти.
Пусть есть массив, состоящий из двух
отсортированных кусочков длины $n$ и $m$ соответственно.
И есть буфер, размером больше, чем  $(n + m)$,
хранящий элементы, значения которых можно менять местами внутри массива, но нельзя потерять.
Отсортируем массив. Для этого произведем merge, используя буфер.
Но, чтобы не потерять значения, хранящиеся в нем, будем менять элементы местами из массива и буфера.
Таким образом окажется, что все элементы первого массива будут в отсортированном порядке лежать в буфере,
а часть элементов из буфера переместится на место элементов массива.
Теперь перенесем все элементы массива обратно и получим отсортированный массив на том же самом месте.

Теперь перейдем к задаче сортировки. Разобьем исходный массив на две части.
Использую определенную выше функцию merge упорядочим вторую половину массива,
используя первую, как буфер, на основе сортировки merge sort.
Затем упорядочим первую четверть массива, использую вторую четверть, как буфер.
Таким образом мы получили упорядоченную первую четверть и вторую половину, а во второй
четверти элементы лежат в произвольном порядке.

Теперь объединим первую четверть и вторую половину массива с помощью операции merge, но записывать
результа начнем с первого элемента второй четверти. И опять же вместо непосредственного
присовения будем обменивать элементы местами. Таким образом получим, что
в первой четверти окажутся все элементы второй четверти, а первая четверть и вторая половина буду
упорядочены и образуют три четверти массива в конце. Теперь в начале массива у нас осталась
одна четверть неупорядоченных элементов. А значит мы можем повторить только что проведенную оперцаю
и досортирвать еще одну восьмую. Так можно продолжать до тех пор, пока не отсортированным останется один элемент массива,
Теперь просто найдем его место в массиве и проведем вставку.
Таким образом мы отсортировали массив с помощью merge сортировки, используя константую память.

\end{document}
