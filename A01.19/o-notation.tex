\documentclass[11pt, oneside]{article}
\usepackage{amsmath}
\usepackage{latexsym}
\usepackage{geometry}
\geometry{letterpaper}
\usepackage[parfill]{parskip}
\usepackage{amssymb}

\usepackage[T2A]{fontenc}
\usepackage[utf8]{inputenc}
\usepackage[russian]{babel}

\parindent=1.25cm

\title{В чем разница между O(f) и o(f)}

\begin{document}
\maketitle

Если быть кратким, то запись $f(n) = O(g(n))$ означает, что функция $f$ ограничена сверху $g$.
При этом, если допустить, что $g$ после некоторого $n$ всегда отлична от нуля,
а обычно с такими функциями вы будете работать в теории алгоритмов,
то определение эквивалентно следующему пределу

\begin{equation*}
    \exists C \in \mathbb{R}^{+} \lim_{n \to \infty } \sup \frac{f(n)}{g(n)} < C
\end{equation*}

В свою очередь запись $f(n) = o(g(n))$ означает, что функция $g$ доминирует $f$.
Или иными словами $f$ бесконечно мала относительно $g$ на бесконечности.
При этом, если допустить, что $g$ после некоторого $n$ всегда отлична от нуля,
то определение эквивалентно следующему пределу

\begin{equation*}
    \lim_{n \to \infty } \frac{f(n)}{g(n)} = 0
\end{equation*}
Вообще говоря $o-$маленькое имеет более строгое условие.

Еще немного о обозначениях. Когда мы пишем, к примеру, $o(f(n)) = O(f(n))$ то подразумеваем,
что любая функция доминируемая сверху $f$ также является ограниченной сверху $f$.
Или вот еще пример,  $O(n^2) = O(n^3)$, любая функция ограниченная сверху функцией $n^2$
также ограничена $n^3$. В обратную же сторону эту утверждение не верно. Короче говоря, все не так сложно.


\end{document}
