\documentclass[10pt, oneside]{article}
\usepackage{amsmath}
\usepackage{latexsym}
\usepackage{geometry}
\geometry{letterpaper}
\usepackage[parfill]{parskip}
\usepackage{amssymb,amsthm,amsmath}


\usepackage[T2A]{fontenc}
\usepackage[utf8]{inputenc}
\usepackage[english, russian]{babel}

\parindent=1.25cm

\theoremstyle{definition}
\newtheorem{problem}{}
\newtheorem{definition}{}

\title{О-нотация, упражнения}

\begin{document}
\maketitle

\begin{definition}
$f(n) = O(g(n))$, если
$\exists C > 0 \; \exists N > 0 \; \forall n \geqslant N : |f(n)| \leqslant C |g(n)|$
\end{definition}

\begin{definition}
$f(n) = o(g(n))$, если
$\forall C > 0 \; \exists N > 0 \; \forall n \geqslant N : |f(n)| < C |g(n)|$
\end{definition}

\begin{definition}
$f(n) = \Omega(g(n))$, если $g(n) = O(f(n))$.
\end{definition}


\begin{definition}
$f(n) = \omega(g(n))$, если $g(n) = o(f(n))$.
\end{definition}

\begin{definition}
$f(n) = \Theta(g(n))$, если $f(n) = O(g(n))$ и $f(n) = \Omega(g(n))$.
\end{definition}


\section{Задачи}

\begin{problem}
Используя определения, докажите следующие свойства:
    \begin{enumerate}
        \item $O(C \cdot f(n)) = O (f(n))$, $C \in \mathbb{R}^{+}$
        \item $o(f(n)) + o(f(n)) = O(f(n))$
        \item $o(f(n)) \cdot O(f(n)) = o(f(n))$
    \end{enumerate}
\end{problem}

\begin{problem}
Покажите, что, если $f(n) = o(g(n))$, то $f(n) = O(g(n))$. Верно ли это в обратную сторону?
\end{problem}

\begin{problem}
Для данных пар функций выясните их связь в терминах
$O$-, $o$-, $\omega$-, $\Omega$-, $\Theta$-обозначений.
\begin{enumerate}
    \item $f(n) = n^{1/2}, \quad g(n) = n^{2/3}$
    \item $f(n) = 100n + \ln n, \quad g(n) = n + (\ln n)^2$
    \item $f(n) = \ln(n)^{\ln n}, \quad g(n) = \frac{n}{\ln(n)}$
\end{enumerate}
\end{problem}

\begin{problem}
Приведите примеры функций $f(n)$ и $g(n)$, таких что
\begin{enumerate}
    \item $f(n) = O(g(n)), f(n) = \Omega(g(n))$
    \item $f(n) = O(g(n)), f(n) \neq \Omega(g(n))$
    \item $f(n) = O(g(n)), f(n) \neq o(g(n))$
\end{enumerate}
\end{problem}

\begin{problem}
Существуют ли такие функции $f(n)$ и $g(n)$, что $f = o(g(n))$ и $f =
\omega(g(n))$? Приведите пример или докажите, что таких функций нет.
\end{problem}

\begin{problem}
Приведите пример функций $f(n)$ и $g(n)$, таких что $f(n) = \Theta(g(n))$ и
$2^{f(n)} = o(2^{g(n)})$ или докажите, что таких функций не существует.
\end{problem}

\end{document}
