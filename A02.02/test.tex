\documentclass[10pt, oneside]{article}
\usepackage{amsmath}
\usepackage{latexsym}
\usepackage{geometry}
\geometry{letterpaper}
\usepackage[parfill]{parskip}
\usepackage{amssymb,amsthm,amsmath}


\usepackage[T2A]{fontenc}
\usepackage[utf8]{inputenc}
\usepackage[english, russian]{babel}

\parindent=1.25cm

% ExCounter:
\newcounter{ExCounter}
\setcounter{ExCounter}{1}

% DefCounter:
\newcounter{DefCounter}
\setcounter{DefCounter}{1}


\newenvironment{exercise}
    {\textbf{Упражнение \arabic{ExCounter}.}}
    {\par\stepcounter{ExCounter}}

\newenvironment{solution}
    {\textit{Решение.}}
    {\par}

\newenvironment{remark}
    {\textit{Замечение.}}
    {\par}


\title{Тест}

\begin{document}
\maketitle

\begin{exercise}
    Дан массив чисел размера $n$. Предложите алгоритм, определяющий есть ли в этом массиве два числа,
    в сумме дающие 100, со сложностью $n \log n$.
\end{exercise}
\begin{solution}
    Отсортируем исходный массив. Затем поставим указатель на первый и последний элементы.
    Если сумма двух текущих элементов больше 100, то будем двигать правый указатель,
    если меньше, то двигать левый указатель, иначе мы нашли искомую пару.
    Алгоритм продолжает работу, пока указатели не встретятся.
\end{solution}

\begin{exercise}
    Студент реализовал рекурсивный алгоритм, который в в лексикографическом порядке выводит на экран
    все строки длиной $n$ состоящие из $0$ и $1$. При этом максимальная глубина рекурсии равна
    $\Theta(n)$. Объясните, почему студенту не стоит об этом беспокоиться.
\end{exercise}
\begin{solution}
    Размер вывода равен $N = n \cdot 2^n$, а значит  глубина рекусии  равна $log N$.
    Это значит, что для $n$, которое подразумевает разумное время исполнения алгоритма,
    стек не переполнится.
\end{solution}

\begin{exercise}
    Верно ли, что  $f(n+1) = O(f(n))$. Ответ обоснуйте.
\end{exercise}
\begin{solution}
    Для этой задачи существует очевидный контрпример, который не будет расскрыт,
    так как такая же задача попала вам в домашнее задание.
\end{solution}

\begin{exercise}
    Докажите, что $2^n = O(n!)$.
\end{exercise}
\begin{solution}
    Для решения можно применить очень грубую оценку $n! > 2^n$.
\end{solution}

\end{document}
