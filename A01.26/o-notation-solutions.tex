\documentclass[10pt, oneside, notitlepage]{article}
\usepackage{amsmath}
\usepackage{latexsym}
\usepackage{geometry}
\geometry{letterpaper}
\usepackage[parfill]{parskip}
\usepackage{amssymb,amsthm,amsmath}


\usepackage[T2A]{fontenc}
\usepackage[utf8]{inputenc}
\usepackage[english, russian]{babel}

\parindent=1.25cm

% ExCounter:
\newcounter{ExCounter}
\setcounter{ExCounter}{1}

% DefCounter:
\newcounter{DefCounter}
\setcounter{DefCounter}{1}


\newenvironment{definition}
    {\textbf{Определение \arabic{DefCounter}.}}
    {\par\stepcounter{DefCounter}}

\newenvironment{exercise}
    {\textbf{Упражнение \arabic{ExCounter}.}}
    {\par\stepcounter{ExCounter}}

\newenvironment{solution}
    {\textit{Решение.}}
    {\par}

\newenvironment{remark}
    {\textit{Замечение.}}
    {\par}


\title{Упражнения}

\begin{document}

\begin{exercise}
    Известно, что $f(n), g(n) > 0$, используя определение,
    докажите следующее свойство $\max(f(n), g(n)) = \Theta(f(n) + g(n))$.
\end{exercise}
\begin{solution}
    Разобьем доказательство на две части.
    \begin{enumerate}
        \item Заметим, что $\max(f(n), g(n))$ можно оценить сферху, как $f(n) + g(n)$,
            т.к. $f(n), g(n) > 0$, следовательно $\max(f(n), g(n)) = O(f(n) + g(n))$.
        \item В свою очередь $f(n) + g(n)$ можно оценить сферху, как $2 \cdot \max(f(n), g(n))$,
            т.к. $f(n), g(n) > 0$, следовательно $f(n) + g(n) = O(f(n) + g(n))$.
    \end{enumerate}
    Из этого по определению вытекает то, что и нужно доказать.
\end{solution}


\begin{exercise}
    Покажите, что $\ln n! = \Theta(n \ln n)$.
\end{exercise}
\begin{solution}
    Наиболее популярный вид записи формулы Стирлинга выглядит следующим образом
    \begin{align*}
        \ln(n!) = n \ln n - n + O(\ln n),
    \end{align*}
    из чего вытекает уже нужная оценка $\ln n! = \Theta(n \ln n)$.
    Упражнение расчитано на то, чтобы вы покапались в теории и узнали про этот факт,
    если вам еще не прочитали его в курсе математического анализа.
\end{solution}

\begin{exercise}
    Пусть $f(n) = [\ln n]!$. Существует ли полином $p(n)$, что $f(n) = O(p(n))$?
\end{exercise}
\begin{solution}
    Исходя из формулы стирлинга справедлива следующая оценка $n! > \Big(\frac{n}{e}\Big)^{n}$.
    В свою очередь $[\ln n] > \ln \frac{n}{e}$. Тогда
    \begin{align*}
        [\ln n]! > \Big(\frac{\ln\frac{n}{e}}{e}\Big)^{\ln\frac{n}{e}}.
    \end{align*}
    Рассмотрим следующий предел
    \begin{align*}
        \lim_{n \to \infty} \frac{
            \frac{n}{e} \cdot n^k
        } {
            \Big(\ln\frac{n}{e}\Big)^{\ln\frac{n}{e}}
        } =
        \lim_{n \to \infty} e^{p} \text{, где }
        p = (k + 1) \cdot n - 1 - \ln\big(\ln(\frac{n}{e})\big)(\ln n - 1)
    \end{align*}
    Заметим, что
    \begin{align*}
        p > (k + 1) \cdot \ln n - 1 - \ln\big(\ln(\frac{n}{e})\big)(\ln n - \frac{1}{k+1}) = \\
        (k + 1)\big(\ln n - \frac{1}{k + 1}\big)\Big(1 - \ln\big(\ln(\frac{n}{e})\big)\Big).
    \end{align*}
    Очевидно, что $p \to - \infty$ при $n \to \infty$. А значит такого многочлена не существует,
    так как предел стремится к $0$ при любом $k$.



\end{solution}

\begin{exercise}
    Верно ли для всех произвольных $f(n), g(n) > 0$, что, если $f(n) = O(g(n))$,
    то и $\ln(f(n)) = O(\ln(g(n)))$.
\end{exercise}
\begin{solution}
    Утверждение неверно, очевидно, что ${e^{-n^2}} = O(e^{-n})$,
    однако, $n^2 \neq O(n)$.
\end{solution}

\begin{exercise}
    Докажите, что $(\ln n)^k = O(n^{\varepsilon})$, где $\varepsilon > 0$, a $k \in \mathbb{N}^{+}$.
\end{exercise}
\begin{solution}
    Рассмотрим следующий предел $\lim_{n \to \infty} \frac{(\ln n)^k}{n^{\varepsilon}}$.
    Для его вычисление воспользуемя правилом Лопиталя. Применим его один раз
    \begin{align*}
        \lim_{n \to \infty} \frac{
            (\ln n)^k
        }{
            n^{\varepsilon}
        } =
        \lim_{n \to \infty} \frac{
            \frac{1}{n} (\ln n)^{k - 1}
        }{
            n^{\varepsilon - 1}
        } =  \lim_{n \to \infty} \frac{
            (\ln n)^{k - 1}
        }{
            n^{\varepsilon}
        }.
    \end{align*}
    Мы уменьшили степень лограрифма, но при это знаменатель не изменился.
    Таки оразом, применив правило $k$ раз, получим следующий предел
    $\lim_{n \to \infty} \frac{1}{n^{\varepsilon}}$.
    А из этого уже следует необходимый нам факт.
\end{solution}

\end{document}
