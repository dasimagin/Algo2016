\documentclass[10pt, oneside]{article}
\usepackage{amsmath}
\usepackage{latexsym}
\usepackage{geometry}
\geometry{letterpaper}
\usepackage[parfill]{parskip}
\usepackage{amssymb,amsthm,amsmath}


\usepackage[T2A]{fontenc}
\usepackage[utf8]{inputenc}
\usepackage[english, russian]{babel}

\parindent=1.25cm

% ExCounter:
\newcounter{ExCounter}
\setcounter{ExCounter}{1}

% DefCounter:
\newcounter{DefCounter}
\setcounter{DefCounter}{1}


\newenvironment{definition}
    {\textbf{Определение \arabic{DefCounter}.}}
    {\par\stepcounter{DefCounter}}

\newenvironment{exercise}
    {\textbf{Упражнение \arabic{ExCounter}.}}
    {\par\stepcounter{ExCounter}}

\newenvironment{solution}
    {\textit{Решение.}}
    {\par}

\newenvironment{remark}
    {\textit{Замечение.}}
    {\par}


\title{Упражнения}

\begin{document}
\maketitle


\begin{definition}
    $f(n) = O(g(n))$,
    если $\exists C > 0$, $n_{0}$: $\forall n > n_{0}$ $|f(n)| \leq C \cdot |g(n)|$.
\end{definition}

\begin{definition}
    $f(n) = o(g(n))$,
    если $\forall C > 0$  $\exists n_{0}:$ $\forall n > n_{0}$ $|f(n)| < C \cdot |g(n)|$.
\end{definition}


\begin{exercise}
    Известно, что $f(n), g(n) > 0$, используя определение,
    докажите следующее свойство $\max(f(n), g(n)) = \Theta(f(n) + g(n))$.
\end{exercise}
% \begin{solution}
% \end{solution}


\begin{exercise}
    Покажите, что $\ln n! = \Theta(n \ln n)$.
\end{exercise}
\begin{remark}
    Если упражнение длительное время вам не сдается, то подумайте, знаете ли вы какие-либо
    асимптотические приближения для n!, если нет, то прочитайте про них.
\end{remark}
% \begin{solution}
% \end{solution}

\begin{exercise}
    Пусть $f(n) = [\ln n]!$. Существует ли полином $p(n)$, что $f(n) = O(p(n))$?
\end{exercise}
% % \begin{solution}
% % \end{solution}

\begin{exercise}
    Верно ли для всех произвольных $f(n), g(n) > 0$, что, если $f(n) = O(g(n))$,
    то и $\ln(f(n)) = O(\ln(g(n)))$.
\end{exercise}
% \begin{solution}
% \end{solution}

\begin{exercise}
    Докажите, что $(\ln n)^k = O(n^{\varepsilon})$, где $\varepsilon > 0$, a $k \in \mathbb{N}^{+}$.
\end{exercise}
% \begin{solution}
% \end{solution}


\end{document}
