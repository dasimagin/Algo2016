\documentclass[10pt, oneside]{article}
\usepackage{amsmath}
\usepackage{latexsym}
\usepackage{geometry}
\geometry{letterpaper}
\usepackage[parfill]{parskip}
\usepackage{amssymb,amsthm,amsmath}


\usepackage[T2A]{fontenc}
\usepackage[utf8]{inputenc}
\usepackage[english, russian]{babel}

\parindent=1.25cm

% ExCounter:
\newcounter{ExCounter}
\setcounter{ExCounter}{1}

% DefCounter:
\newcounter{DefCounter}
\setcounter{DefCounter}{1}


\newenvironment{definition}
    {\textbf{Определение \arabic{DefCounter}.}}
    {\par\stepcounter{DefCounter}}

\newenvironment{exercise}
    {\textbf{Упражнение \arabic{ExCounter}.}}
    {\par\stepcounter{ExCounter}}

\newenvironment{solution}
    {\textit{Решение.}}
    {\par}


\title{Упражнения}

\begin{document}
\maketitle


\begin{definition}
    $f(n) = O(g(n))$,
    если $\exists C > 0$, $n_{0}$: $\forall n > n_{0}$ $|f(n)| \leq C \cdot |g(n)|$.
\end{definition}

\begin{definition}
    $f(n) = o(g(n))$,
    если $\forall C > 0$  $\exists n_{0}:$ $\forall n > n_{0}$ $|f(n)| < C \cdot |g(n)|$.
\end{definition}


\begin{exercise}
Используя определения, докажите следующие свойства:
    \begin{enumerate}
        \item $O(C \cdot f(n)) = O (f(n))$, $C \in \mathbb{R}^{+}$
        \item $o(f(n)) + o(f(n)) = O(f(n))$
        \item $o(f(n)) \cdot O(f(n)) = o(f(n))$
    \end{enumerate}
\end{exercise}
\begin{solution}
    Для доказательства достаточно раскрыть определение.
\end{solution}


\begin{exercise}
    Покажите, что, если $f(n) = o(g(n))$, то $f(n) = O(g(n))$. Верно ли это в обратную сторону?
\end{exercise}
\begin{solution}
    Ответ следует из определений. Очевидно, что если $\forall C > 0$ верно,
    что $|f(n)| < C \cdot |g(n)|$,то автоматически выполняется условие, что $f(n) = O(g(n))$.
    В обратну сторону это, конечно, не верно.
    Заметим, что $f(n) = O(f(n))$,
    однако $f(n)$ не может быть бесконечна мала относительно себя же,
    поэтому $f(n) \neq o(f(n))$.
\end{solution}

\begin{exercise}
    Для данных пар функций выясните их связь в терминах
    $O$-, $o$-, $\omega$-, $\Omega$-, $\Theta$-обозначений.
    \begin{enumerate}
        \item $f(n) = n^{\frac{1}{2}}, \quad g(n) = n^{\frac{2}{3}}$
        \item $f(n) = 100n + \ln n, \quad g(n) = n + (\ln n)^2$
        \item $f(n) = \ln(n)^{\ln n}, \quad g(n) = \frac{n}{\ln(n)}$
    \end{enumerate}
\end{exercise}
\begin{solution}
    \begin{enumerate}
        \item Рассмотрим предел
            \begin{align*}
                \lim_{n \to \infty} \frac{f(n)}{g(n)} =
                \lim_{n \to \infty} \frac{n^{\frac{1}{2}}}{n^{\frac{2}{3}}} =
                \lim_{n \to \infty} n^{- \frac{1}{6}} = 0,
            \end{align*}
        а это значит, что $f(n) = o(g(n))$.
        \item Очевидно, что $f(n)$ и $g(n)$
        ограничены снизу $\Omega(n)$. Рассмотрим следующий предел
        \begin{align*}
            \lim_{n \to \infty} \frac{(\ln n)^{k}}{n}, \quad k \in \mathbb{N}^{+}.
        \end{align*}
        Применив правило Лопиталя $k$ раз, получим
        \begin{align*}
            \lim_{n \to \infty} \frac{(\ln n)^{k}}{n} =
            \lim_{n \to \infty} k \cdot \frac{(\ln n)^{k - 1}}{n} =
            \dots =
            \lim_{n \to \infty} \frac{k!}{n} =
            0.
        \end{align*}
        Из этого следует, что $(\ln n)^{k} = o(n)$, где $k \in \mathbb{N}^{+}$,
        а значит $f(n)$ и $g(n)$ ограничены сверху $O(n)$.
        Из чего делаем вывод, что обе функции ограничены $\Theta(n)$.

        \item Опять же рассмотрим предел \begin{equation*} \end{equation*}
        \begin{align*}
            \lim_{n \to \infty}
            \frac {
                \frac{n}{\ln n}
            } {
                (\ln n)^{\ln n}
            }
            \leq
            \lim_{n \to \infty}
            \frac{
                n
            } {
                (\ln n)^{\ln n}
            }
        \end{align*}
        Сделаем преобразование и получим,
        \begin{align*}
            \lim_{n \to \infty}
            \exp \Big(
                \ln n - \ln \big( (\ln n)^{\ln n} \big)
            \Big) =
            \lim_{n \to \infty}
            \exp \Big(
                \ln n \cdot \big( 1 - \ln \ln n \big)
            \Big) =
            0,
        \end{align*}
        так как выражение в степени стремится к $- \infty$.
        А это значит, что $g(n) = o(f(n))$.
    \end{enumerate}
\end{solution}

\begin{exercise}
    Приведите примеры функций $f(n)$ и $g(n)$, таких что
    \begin{enumerate}
        \item $f(n) = O(f(n))$, $f(n) = \Omega(g(n))$
        \item $f(n) = O(g(n))$, $f(n) \neq \Omega(g(n))$
        \item $f(n) = O(g(n))$, $f(n) \neq o(g(n))$
    \end{enumerate}
\end{exercise}
\begin{solution}
    Это упражнение проверяет ваше понимание определений, возможны следующие ответы:
    \begin{enumerate}
        \item $f(n) = 100n + \ln n$ и $g(n) = n$.
        \item $f(n) = n$ и $g(n) = n^2$.
        \item $f(n) = 2n$ и $g(n) = n + 3$.
    \end{enumerate}
\end{solution}

\begin{exercise}
Существуют ли такие функции $f(n)$ и $g(n)$, что $f = o(g(n))$ и $f =
\omega(g(n))$? Приведите пример или докажите, что таких функций нет.
\end{exercise}
\begin{solution}
    Очевидно, что таких функций нет. Иначем можно получить следующее неравество,
    если посдавить в определения для $o$ и $\omega$ $C = 1$ для
    \begin{align*}
        f(n) < g(n) < f(n).
    \end{align*}
\end{solution}

\begin{exercise}
Приведите пример функций $f(n)$ и $g(n)$, таких что $f(n) = \Theta(g(n))$ и
$2^{f(n)} = o(2^{g(n)})$ или докажите, что таких функций не существует.
\end{exercise}
\begin{solution}
    $f(n) = n^2$ и $g(n) = n^2 + n$.
\end{solution}

\end{document}
